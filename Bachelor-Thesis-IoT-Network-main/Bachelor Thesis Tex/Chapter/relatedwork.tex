\chapter{Related Work}
This chapter presents an overview of the literature surrounding IoT monitoring systems, IoT in the art industry, LTE, and IoT security. Showing the current standards and practices in the art transport monitoring industry, and comparing this thesis to them.

\section{Artwork Tracking}
\cite{Ramesh2017, Subahi2020, Rahman2020, Karim2018} discuss IoT monitoring systems designed for tracking temperature and humidity. \cite{Subahi2020} focuses on monitoring the temperature within greenhouse environments, aiming to improve agriculture production. \cite{Rahman2020} is dedicated to monitoring the environmental data within a data center, while \cite{Karim2018} specializes in monitoring the conditions within food storage facilities.

\cite{Shah2016, Prasanth2021, Hinostroza2022, Alsuhly2018, Fort2022} delve into IoT monitoring solutions tailored for buildings within the art industry. \cite{Shah2016, Prasanth2021, Hinostroza2022} are dedicated to monitoring Heritage Artefacts, while \cite{Fort2022} focuses on ancient wood structures.
Additionally, \cite{Fort2022} not only monitors artwork within local museum settings but also provides remote monitoring capabilities for artwork on the move. Furthermore, \cite{Zhang2021} addresses the need for a transport monitoring system aimed at protecting cultural relics during transit.

\section{Long Term Evolution}
Many IoT systems require connectivity to the internet through a cellular network.
\cite{Dian2020} talks about LTE-M and narrow-band IoT and provides an overview of their evolution.
\cite{Ratasuk2015} provides an overview of LTE enhancements for machine-type communications, alongside an analysis of LTE capacity for machine traffic and an evaluation of device battery life.

\section{IoT Security}
Interestingly, not a lot of these papers address security concerns.
In contrast, \cite{Noor2019, Mahmoud2015} delves into the security aspects of IoT devices, showing vulnerabilities and corresponding solutions for each of the three layers of the IoT device architecture.
\cite{Garadi2020} provides a comprehensive survey of machine learning and deep learning methods that can be used to enhance the security of IoT systems, discussing the opportunities, advantages, and challenges of these methods. \cite{Neshenko2019} focuses on the vulnerabilities, providing a classification of related surveys, a unique taxonomy of IoT vulnerabilities, and a first look at Internet-scale IoT exploitations.

\section{Discussion}

\begin{table}[H]
    \centering
    \begin{tabular}{|c|c|} \hline 
        \textbf{Paper} & \textbf{Focus Area}\\
        \hline
        [19, 29, 18, 11] & IoT Monitoring Systems\\ \hline 
        [23, 3, 10, 1, 6] & IoT in Art Industry\\ \hline 
        [38] & Transport Monitoring\\ \hline 
        [4, 20] & LTE\\ \hline 
        [9, 15, 7, 16] & IoT Security\\ \hline 
        This Thesis & Secure IoT Transport Monitoring System using LTE\\ \hline
    \end{tabular}
    \caption{Paper Overview}
    \label{tab:Paper Overview}
\end{table}

As shown in Table 2.1, prior research has concentrated on various aspects of IoT, including monitoring systems, its application in the art industry and transportation, LTE utilization for IoT, and security concerns regarding IoT devices. This thesis delves into these domains, emphasizing the need for precise environmental monitoring to ensure the safety and preservation of artworks during transit.