\chapter{Introduction}
The art market is a diverse ecosystem involving various stakeholders,  from artists and collectors to auction houses, art dealers, and various intermediaries such as promoters, preservers, and curators. Despite this diversity, the core of the art world revolves around art objects, ranging from traditional sculptures and canvases to more unconventional pieces like bananas affixed to walls. Museums often showcase artworks from private collections, relying on the involvement of reliable logistics partners to safely transport and preserve these valuable pieces while monitoring environmental conditions such as temperature and humidity. 

\section{Motivation}
Integrating Internet of Things (IoT) sensors into the logistics of art transportation presents a significant opportunity to enhance security and introduce standardization into what is currently an unregulated market.
As such, it requires the implementation of a cloud-based service infrastructure, supporting IoT security and communication during art transport. This communication shall be enabled
through Long Term Evolution (LTE) and secured through a Virtual Private Network (VPN). 

During transportation, the environmental conditions surrounding the artwork are continuously monitored, with real-time updates broadcasted to a cloud gateway. To enable this, each truck has a designated ST-Board, which uses sensors to measure the environmental data and to facilitate communication from the truck to the cloud-based gateway.

The primary focus of this thesis lies in the communication between transportation and museum entities, with the use of ST-board-based technology. Therefore, the primary goal is to develop and implement a prototype artwork sensing and logging system that can deliver precise and accurate measurements to all involved stakeholders.

\section{Description of Work}
This Bachelor Thesis consists of three stages:

\textbf{Research:}
The initial stage is dedicated to research, where related works are reviewed and essential background knowledge is acquired. This stage involves familiarizing with the artwork use case and defining requirements for both sensing and communication. Existing solutions, approaches, and their components will be evaluated, and their applicability, performance, and security will be considered. This stage also necessitates research into LTE communication and VPN, as they form the basis of communication. The goal is to gain sufficient insight into the theoretical aspects of this work, enabling the conceptualization of feasible approaches and improvements over existing solutions. Additionally, the capabilities of the ST-board, the available sensors, and potential additions will be explored.

\textbf{Design:}
The second stage uses the knowledge gained from the first stage to design a prototype and plan experiments to thoroughly evaluate the logging functionality and behavior, as well as the communication paths and their stability. In this stage, the artwork transportation scenario with its actors is shown, the technical components are listed and described and an overview of the system is presented.

\textbf{Implementation and Evaluation:}
The final stage involves implementing the prototype based on the design, considerations, and limitations determined in the first two stages. This stage also includes conducting experiments to assess the reliability of the implementation and evaluating its overall reliability, limitations, and potential expansions.

In addition to these stages, this thesis will address the following research questions:

\begin{itemize}
    \item What are the current standards and practices in the art transport industry for monitoring environmental conditions?
    \item How effective is the prototyped system in real-world scenarios?
\end{itemize}

\section{Thesis Outline}
This thesis is structured as follows:

Chapter two delves into a review of various approaches aiming to address the artwork monitoring problem.
In chapter three, the theoretical knowledge necessary for understanding the thesis is presented.
Chapter four explains the scenario, including its system design and testing procedures.
The implementation of the system is shown in chapter five.
Chapter six conducts testing and evaluation of the implemented system.
Finally, chapter seven provides a concise summary of the thesis and gives an outlook for future work.