\chapter*{Zusammenfassung}
\addcontentsline{toc}{chapter}{Zusammenfassung}
\selectlanguage{german}

Diese Arbeit untersucht die Integration von Internet of Things (IoT)-Sensoren in die Logistik von Kunsttransporten, um die Sicherheit zu erhöhen und eine Standardisierung in einem unregulierten Markt einzuführen. Sie beinhaltet die Implementierung einer Cloud-basierten Service-Infrastruktur, die IoT-Sicherheit und -Kommunikation während des Kunsttransports durch Long Term Evolution (LTE) unterstützt und über ein Virtual Private Network (VPN) gesichert ist. Der Schwerpunkt liegt auf der Entwicklung eines Prototyps für ein System zur Erfassung und Protokollierung von Kunstwerken, das den am Transport von Kunstwerken beteiligten Akteuren präzise Messungen liefert. Das System soll die Umgebungsbedingungen rund um das Kunstwerk in Echtzeit überwachen und so dessen Sicherheit und Unversehrtheit während des Transports gewährleisten. Die Arbeit beschreibt die Forschungs-, Design-, Implementierungs- und Evaluierungsphasen des Projekts und befasst sich mit den aktuellen Standards und Praktiken im Kunsttransport, der Effektivität des Prototyps und der zukünftigen Arbeit in diesem Bereich.

Das System, das ursprünglich für die Nutzung von Cloud-Diensten und Mobilfunkverbindungen konzipiert war, stiess auf Probleme mit den Netzwerkverbindungen in der Schweiz. Daher wurde eine lokale Lösung unter Verwendung des STMicroelectronics B-L462E-CELL1 Cellular Discovery Kit implementiert. Dieser lokale Prototyp misst Temperatur, Luftfeuchtigkeit und Beschleunigung und zeigt die Daten zur Echtzeitüberwachung direkt auf einem angeschlossenen Laptop. Trotz der Abkehr von der beabsichtigten Fernüberwachung hat der lokale Prototyp seine Robustheit und Zuverlässigkeit in verschiedenen simulierten Transportszenarien bewiesen und damit sein Potenzial für reale Anwendungen im Kunsttransport unter Beweis gestellt. Zukünftige Arbeiten zielen auf die Lösung von Konnektivitätsproblemen und die Implementierung des ursprünglichen Fernüberwachungsdesigns ab.


\chapter*{Abstract}
\addcontentsline{toc}{chapter}{Abstract}
\selectlanguage{english}

This thesis explores the integration of Internet of Things (IoT) sensors into the logistics of art transportation to enhance security and introduce standardization in an unregulated market. It involves the implementation of a cloud-based service infrastructure, supporting IoT security and communication during art transport through Long Term Evolution (LTE) and secured via a Virtual Private Network (VPN). The focus is on developing a prototype artwork sensing and logging system that delivers precise measurements to stakeholders involved in the transportation of artworks. The system aims to monitor environmental conditions surrounding the artwork in real time, ensuring its safety and integrity during transit. The thesis outlines the research, design, implementation, and evaluation stages of the project, addressing current standards and practices in art transport, the effectiveness of the prototyped system, and future work in the field.

Initially designed to utilize cloud services and cellular connectivity, the system faced challenges with network connections in Switzerland. Consequently, a local solution was implemented using the STMicroelectronics B-L462E-CELL1 Cellular Discovery Kit. This local prototype measures temperature, humidity, and acceleration, and logs data directly to a connected laptop for real-time monitoring. Despite the shift from the intended remote monitoring, the local prototype demonstrated robustness and reliability in various simulated transportation scenarios, proving its potential for real-world applications in art transport. Future work aims to resolve connectivity issues and implement the initial remote monitoring design.