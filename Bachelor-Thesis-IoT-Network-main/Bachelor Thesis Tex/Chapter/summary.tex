\chapter{Summary and Conclusions}

The purpose of this chapter is to wrap up the thesis by providing a summary, drawing conclusions, and discussing potential future work. These discussions are based on the insights gained during the development process and the identified limitations of the system.

\section{Summary}
This thesis introduces a system designed for monitoring artwork during transportation. The concept involves the integration of an IoT device onto the artwork, tasked with capturing and storing critical environmental data like temperature, humidity, and acceleration to ensure the safety of the artwork. The board assesses this recorded data against predefined thresholds and notifies stakeholders if these thresholds are reached. The measured data is stored in the cloud and distributed to various stakeholders via the MQTT protocol. To demonstrate the feasibility of the proposed system, a prototype was developed as part of this thesis.

The first step involved gaining the theoretical knowledge for this project through a literature review. The thesis provided the theoretical background on IoT, microcontrollers, wireless communication technologies, cloud computing, MQTT, and their application in artwork tracking, along with addressing IoT security concerns.

The system design includes a description of an artwork monitoring scenario, an overview of the involved actors, a list of the technical components, an overview of the system architecture, and a testing methodology to evaluate the system. The system contains a monitoring application for the stakeholders to monitor the artwork during transport, and the IoT device which measures the environmental data. This data is then transmitted to the cloud, which not only distributes it to stakeholders but also stores it in a database for future reference and analysis.

The implementation of the prototype followed the outlined system design, with the B-L462E-CELL1 board serving as the IoT device. The first step was to enable the HTS221 sensor to measure temperature and humidity and the LSM303AGR sensor to measure acceleration. After that, logging of the measured data was implemented. The next step was to establish a cellular network connection, which unfortunately was not possible. Due to the network issues, the designed system was no longer viable, necessitating the implementation of a local solution. Instead of using a monitoring application, the artwork has to be monitored locally. For this purpose, a tool to monitor and analyze the data has been implemented in Python.

The system was evaluated in four simulated artwork monitoring scenarios. The first scenario tested the longevity of the system, by making it run for 28 hours. The second scenario focused on testing communication between the board and laptop, along with measurements in a dynamic environment. In the third scenario, speed was introduced during a car ride, representing a more realistic scenario. The fourth scenario was a hybrid test, involving transportation by walking, bus, and train to assess transitions between different transport modes.

\section{Conclusion}
A comprehensive exploration of the standards and practices within the artwork industry has been conducted, giving insight into the diverse applications of IoT devices for artwork monitoring across various settings and environments.
The prototype for monitoring artwork during transportation was successfully designed. However, due to network connection issues, the planned remote implementation was not feasible, leading to a local implementation instead. The lack of documentation for the board made the implementation challenging. The local setup involves plotting the measured data to a laptop connected to the board, allowing for continuous local monitoring of the artwork. The data can be saved locally for future analysis. The effectiveness of the local prototype was validated through three simulations of artwork transportation, demonstrating its reliability in real-world scenarios.

\section{Future Work}
The goal of a remote monitoring system using the cloud was not reached, due to network connection issues, therefore these issues should be resolved and the initial design should be implemented in the future. Once that is implemented there are some improvements which can be done. To minimize the cost of data transmission, save the data locally and transmit it only when necessary, such as when a threshold is reached or the memory of the board is approaching its capacity. Another improvement could be to connect an air conditioner or a heater to the board, to regulate the temperature automatically once the predefined thresholds are met.

Due to the mentioned issues a local prototype to monitor artwork during its transportation was implemented. One improvement would be to automatically save the plots during the monitoring, instead of having to save them manually at the end of the transportation. Another one would be to add an alarm, once a threshold is reached. The alarm could be a print to the console or an audible alert to warn the user to act accordingly. Lastly, a script that automates all the necessary steps for the system to work should be implemented, simplifying user interaction to just running the script.